
\documentclass[11pt]{article}

\usepackage[utf8]{inputenc} % Required for inputting international characters
\usepackage[T1]{fontenc} % Output font encoding for international characters

\usepackage{mathpazo} % Palatino font
\usepackage{graphicx}
\usepackage{geometry}
\usepackage{float}
\usepackage{amsmath}
\usepackage{tabu}
\usepackage{array}
\usepackage{cellspace}
\usepackage[table]{xcolor}
\usepackage{multirow}
\usepackage{multicol}
\usepackage{placeins}
\usepackage{blindtext}
\renewcommand{\baselinestretch}{1.25}
\geometry{margin=.75 in}

\begin{document}

%----------------------------------------------------------------------------------------
%	TITLE PAGE
%----------------------------------------------------------------------------------------

\begin{titlepage} % Suppresses displaying the page number on the title page and the subsequent page counts as page 1
	\newcommand{\HRule}{\rule{\linewidth}{0.5mm}} % Defines a new command for horizontal lines, change thickness here
	
	\center % Centre everything on the page
	
	%------------------------------------------------
	%	Headings
	%------------------------------------------------
	
	\textsc{\LARGE Portland State University
}\\[.25cm] % Main heading such as the name of your university/college
	
	\textsc{\Large Department of Electrical and Computer Engineering }\\[1cm] % Major heading such as course name
\includegraphics[width=0.2\textwidth]{psuLOGO.jpg}\\[1cm]	
	\textsc{\LARGE2019  ECE Capstone }\\[0.12cm] % Minor heading such as course title
	\textsc{\LARGE Team 21 }\\[0.12cm]	
		\textsc{\Large Faculty Advisor: Roy Kravitz, M.S. }\\[0.12cm]
		\textsc{\Large Sponsor: Sightline Applications }\\[1cm]
	
	%------------------------------------------------
	%	Title
	%------------------------------------------------
	
	\HRule\\[0.4cm]
	
	{\huge\bfseries UAV Landing Aid - Design Report}\\[0.4cm] % Title of your document
	
	\HRule\\[1.5cm]
	
	%------------------------------------------------
	%	Author(s)
	%------------------------------------------------
	
%	\begin{minipage}{0.4\textwidth}
%		\begin{flushleft}
%			\large
%			\textit{By:}\\
%		 \textsc{Kimball S. Davis} % Your name
%		\end{flushleft}
%	\end{minipage}
%	~
%	\begin{minipage}{0.4\textwidth}
%		\begin{flushright}
%			\large
%			\textit{Professor of Electrical \& Computer Engineering}\\
%		 \textsc{Professor Branimir Pejcinovic, Ph.D.} % Supervisor's name
%		\end{flushright}
%	\end{minipage}
	
	% If you don't want a supervisor, uncomment the two lines below and comment the code above
	{\Large\textit{Authors:}}\\
 	\Large\textsc{Tai Pham}\\
	 \Large\textsc{Kimball S. Davis} % Your name
	
	%------------------------------------------------
	%	Date
	%------------------------------------------------
	
	\vfill\vfill\vfill % Position the date 3/4 down the remaining page
	
	{\large\today} % Date, change the \today to a set date if you want to be precise
	
	%------------------------------------------------
	%	Logo
	%------------------------------------------------
	
	%\vfill\vfill
	%\includegraphics[width=0.2\textwidth]{PSULOGO}\\[1cm] % Include a department/university logo - this will require the graphicx package
	 
	%----------------------------------------------------------------------------------------
	
	\vfill % Push the date up 1/4 of the remaining page
	
\end{titlepage}

%----------------------------------------------------------------------------------------


\section*{Abstract}

\blindtext

\pagebreak

\tableofcontents

\pagebreak

\listoffigures

\pagebreak

\section{Project Overview}
\blindtext
\subsection{Problem Definition}
\blindtext
\subsection{Solution(10,000ft view)}
\blindtext

    \begin{figure}[H]
	\centering	
	\caption{\textit{This is a caption for a blank figure}}	
	\end{figure}

%    \begin{figure}[H]
%	\centering	
%	\includegraphics[width=3.5 in]{Simple_SatR}
%	\caption{\textit{Simple saturable reactor[1]}}	
%	\end{figure}

\section{Results(Technical Detail)}
\subsection{Quadcopter}

\blindtext

    \begin{figure}[H]
	\centering	
	\caption{\textit{This is a caption for a blank figure}}	
	\end{figure}

\subsection{Hardware}

\blindtext

    \begin{figure}[H]
	\centering	
	\caption{\textit{This is a caption for a blank figure}}	
	\end{figure}

\subsection{Software}

\blindtext

    \begin{figure}[H]
	\centering	
	\caption{\textit{This is a caption for a blank figure}}	
	\end{figure}



\section{Conclusion/What's Left to do}

\blindtext


     \begin{figure}[H]
	\centering	
	\caption{\textit{This is a caption for a blank figure}}	
	\end{figure}

\pagebreak

\section{Appendix}
\subsection{I}
\subsection{II}
\subsection{II}
\subsection{IV}
\subsection{V}	

\pagebreak

\section{References}
\begin{enumerate}
\item Mali, P. (1960).\textit{ Magnetic Amplifiers}. New York, NY: John F. Rider.
\item Figure 1. Simple saturable reactor. Reprinted from \textit{Magnetic Amplifiers}(pg. 27), by Paul Mali, 1960, New York, NY, John F. Rider Publisher, Inc.
\item Figure 2. Full-wave self-saturating magnetic amplifier. Reprinted from \textit{Magnetic Amplifiers}(pg. 35), by Paul Mali, 1960, New York, NY, John F. Rider Publisher, Inc.
\end{enumerate}	
\end{document}
